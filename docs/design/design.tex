\documentclass[onecolumn, draftclsnofoot,10pt, compsoc]{IEEEtran}
\usepackage{graphicx}
\usepackage{url}
\usepackage{setspace}
% Set backend as appropriate
\usepackage[style=ieee,backend=biber]{biblatex}
\usepackage{geometry}
\bibliography{references}
\geometry{textheight=9.5in, textwidth=7in}

\def \CapstoneTeamName{		Enteral Feeding Calculator Team}
\def \CapstoneTeamNumber{		60}
\def \GroupMemberOne{			Alison Jones}
\def \GroupMemberTwo{			Parker Okonek}
\def \CapstoneProjectName{		Volume-Based Enteral Feeding Calculator}
%\def \CapstoneSponsorCompany{	Company, Inc}
\def \CapstoneSponsorPerson{		Dr. Judy Davidson}

\def \DocType{
				Design Document
				}
			
\newcommand{\NameSigPair}[1]{\par
\makebox[2.75in][r]{#1} \hfil 	\makebox[3.25in]{\makebox[2.25in]{\hrulefill} \hfill		\makebox[.75in]{\hrulefill}}
\par\vspace{-12pt} \textit{\tiny\noindent
\makebox[2.75in]{} \hfil		\makebox[3.25in]{\makebox[2.25in][r]{Signature} \hfill	\makebox[.75in][r]{Date}}}}
% 3. If the document is not to be signed, uncomment the RENEWcommand below
%\renewcommand{\NameSigPair}[1]{#1}

%%%%%%%%%%%%%%%%%%%%%%%%%%%%%%%%%%%%%%%
\begin{document}
\begin{titlepage}
    \pagenumbering{gobble}
    \begin{singlespace}
        \hfill 
        \par\vspace{.2in}
        \centering
        \scshape{
            \huge CS Capstone \DocType \par
            {\large\today}\par
            \vspace{.5in}
            \textbf{\Huge\CapstoneProjectName}\par
            \vfill
            {\large Prepared for}\par
            {\Large\NameSigPair{\CapstoneSponsorPerson}\par}
            {\large Prepared by }\par
            Group\CapstoneTeamNumber\par
            \CapstoneTeamName\par 
            \vspace{5pt}
            {\Large
                \NameSigPair{\GroupMemberOne}\par
                \NameSigPair{\GroupMemberTwo}\par
            }
            \vspace{20pt}
        }
        \begin{abstract}%temp abstract until doc is complete.
        This document details the design components of the Enteral Feeding Calculator project. The purpose of this project is to increase efficiency and effectiveness among nurses during the tube feeding process. The document covers product importance, usage, stakeholders, and views. Our solution will be an application designed for either mobile device or computer interfaces with potential methods specified below.
        \end{abstract}     
    \end{singlespace}
\end{titlepage}
\newpage
\pagenumbering{arabic}
\tableofcontents
\clearpage






%=======================================================
\section{Overview}
%Alternatively we could just have an overview paragraph and no subsections.
\subsection{Purpose}
The main goal of the project is replace the manual method of calculating volumetric enteral feeding rates with an automated, digital method that results in less errors and takes less time to complete.
This document outlines the development path of the project and details the implementation and design on the calculator.
\subsection{Scope}
The calculator will return adjusted feeding rates to accommodate for missed feeding time for a patient from any variety of factors.
The user experience will focus on presenting a minimal and easy to read design where the user, a nurse, enters in the minimum needed and easiest to derive data to determine the feeding rate and is returned the adjusted safe rate to catch up on daily volume.
\subsection{Intended Audience}
The intended audience of this design document are the sponsors, Dr. Judy Davidson and her team, and the student development team.
The document serves as a reference for the student development team to track the product creation timeline and as a representation of the sponsor's needs and product expectations that is verifiable and referable.
%=======================================================
\section{Definitions}
Enteral Feeding: A form of tube feeding used to deliver the appropriate nutrients to a patient.This is done through either the mouth, esophagus, or directly into the digestive system through an artificial opening. (Not to be confused with intravenous administration of medication!)\\
IRB:  An Institutional Review Board is a group that has received authority to review, deny, approve, request modification to, and monitor biomedical research in its institution.
\newline
\newline
%etc.
%https://www.fda.gov/media/75459/download Significant vs non-sig device

%=======================================================
\section{Project Context}
\subsection{Hardware}
%Our project will require either a phone or computer (depending on platform) for both development and in production.
The software will support either the mobile platform or the desktop platform.
If the project is developed for mobile, it will be able to run on most off the shelf smart phones or tablets.
The project will require both a desktop and mobile device for development when targeting the mobile platform.
If the project is developed for desktop, it can be completely developed using normal desktop hardware and be expected to run on common consumer desktops.
\subsubsection{Desktop Requirements}
\begin{enumerate}
    \item System Architecture: i386 or amd64
    \item Screen Resolution: 1024 x 768 minimum
    \item Input: Keyboard and Mouse, or on-screen keyboard and pointer
\end{enumerate}
\subsubsection{Mobile Requirements}
\begin{enumerate}
    \item System Architecture: ARMv8
    \item Screen Resolution: 720 x 1280 minimum
    \item Input: Pointer and on-screen keyboard or pointer and physical keyboard
\end{enumerate}
\subsection{Software}
Software requirements vary depending upon the chosen platform.
\subsubsection{Desktop Requirements}
\begin{enumerate}
    \item Version Control: Github
    \item Operating System: Windows 7 or greater
    \item UI Toolkit: Qt (other options, see below)
\end{enumerate}
\subsubsection{Mobile Requirements}
\begin{enumerate}
    \item Version Control: Github
    \item Operating System: Android 7.0 or greater, (or iOS 12.0 or greater?)
    \item UI Toolkit: React Native (other options, see below)
\end{enumerate}
%ComputerOS/PhoneOS
%platform
%resulting language
%any libraries (doubt)

%=======================================================
\section{Design Description}
\subsection{Design Stakeholders}
%Judy, Other people we talked to? Basically who is in charge of this project and is giving their time to get back from it.
Dietitians: These are the people who dispense food and check with physicians to ensure the nutrition is sufficient for the patient's needs. 
\newline
Nurses: Nurses are our target user since they will be the ones directly using the application. Nurses conduct all the direct work including setting up feeding bags, doing the current calculations for catch-up feeding rates, and administering food.
\newline
Physicians: Physicians are the people that check the prescription and administration are performed accurately. They effectively oversee the process and verify feeding needs with dietitians.
\newline
\newline
Michael Mercer is a dietitian in charge of dietitian standards.
%decrease time to nutritional goal
\newline
Patricia Graham is a nurse lead providing us lots of insight on how nurses do their jobs and what they might want or expect from our application.
\newline
Dr. Heyland is the person who invented the original feeding rate catch-up chart. We are effectively automating the usage of this chart and turning it into an application. 
\subsection{Design View}
\subsubsection{Users}
%what is the nurse's view?
The enteral feeding calculator is a nurse's tool to easily retrieve an adjusted feeding rate.
For nurses, the end users, using the calculator will be a simple and time-efficient method when compared to a paper table or manual calculation.
The user-facing interface for the calculator will have a minimal number of inputs needed to reliably calculate the adjusted feeding rate to decrease possible input errors and promote common usability principles.

A simple interface lends itself well to both learnability, how easy it is for new users to learn the interface, and efficiency, the speed of task completion when a user has learned the interface.
The interface will also feature either a prominent calculation button or automatic result return field to aid in efficiency and readability.
\subsubsection{Stakeholders}%think people from the various zoom meetings? This is the WHY to Stakeholders section?

\subsection{Design Timeline}
%Gantt Chart goes here
%\includegraphics{gantt}\\
%\begin{center}
%Caption if we need it
%\end{center}

\subsection{Design Viewpoints}
\subsubsection{Context Viewpoint}
%'black box' view from user perspective.
The application should present a more accessible and easier to use option than the current methods of calculating the adjusted volumetric feeding rate.
The user experience should be hassle-free, without any unexpected slow-downs, crashes, or graphical artifacts within the program's control.
%What should they see/ what should be taken into account for this view? (design)
%How can we test this?(analytics)

A select test group of nurses will be surveyed to determine the optimal platform to improve usage rates and satisfaction.
A similar group, or the same group, will also be asked to review paper prototypes of the applications basic UI layout.
Once the interface for the program is developed, it will be tested by the developers, sponsors, and target users to verify the app functions as expected and returns accurate results.

%Why is this view important (rationale)

\subsubsection{Legal Viewpoint}
In September of 2019, the FDA released a set of changes to medical software policies due to the 21st Century Cures Act.
Due to these changes the enteral feeding calculator on a mobile platform would fall under the regulatory category of "mobile medical app" due to its use as an accessory to a regulated medical device \autocite{FDA}.

The enteral feeding tube calculator may not be considered a non-significant risk device under FDA laws though could be approved under the app's IRB.
The app does supplement the functionality of a pump that needs to have a rate of flow entered to properly feed a patient; however, if the calculations done by the app are considered to be a "simple task" as described in the FDA's \textit{Policy for Device Software Functions and Mobile Medical Applications} then this would not be the case.
The application generates a flow rate that is entered into a medical device without outside verification and for patient safety should have its accuracy and error-rate examined by an IRB.

\subsubsection{Composition Viewpoint}
%How the project interacts with itself. How is it composed?
The project will be composed of three distinct pieces: a user-facing UI to collect and report data, a system interface to manage local storage and compatibility, and the internal logic which calculates results from user input and keeps the program state.

The user-facing interface presents an easy to read form input for the minimum required information to calculate a result and passes those values to the internal logic, when a result is created the interface will display this to the user as well.

The system interface manages the constant values between application runs that is not subject to regular change, for example daily target volume or time of feed reset, and allocates required resources from the operating system.
The internal program logic manages the resulting rate and integrates the other two components of the program with itself and each other.
This component will also manage the notifications to the user for the UI, such as missing inputs, invalid inputs, and indicating the new rate has been calculated.
%Why is this setup important (design)

This composition ensures that the UI framework can be replaced in the event of the current choice losing active maintenance or not adapting to the changing software landscape.
It additionally separates the program logic from the interface display and system interfaces, reducing latency from calculation and allowing the program logic to remain consistent and constant regardless of other changes.
%How to test product's composition. ex: can different developers easily pass it around? (analytics)
%Why is this view important (rationale)
%\subsubsection{Interface Viewpoint}
%Are there multiple interfaces, why are they all important and how do they all work together to make one product? (may not be relevant for us!)

\subsection{Design Rationale}
%what we decided to use and bullet list of reasons why.
% maybe see other rationales for this one?

%=======================================================
\section{Methods}

%Parker Sections
%UI
%%%%%%%%%%
% This serves the same purpose as the later section about interface design
%%%%%%%%%%
%\subsection{User Interface Framework}
%\subsubsection{Approach}
%The uncertainty of a platform requires that the user interface framework is %multi-platform.
%Assuming a multi-platform or desktop implementation, Qt, supports a variety of %platforms with similar or identical interfaces with the underlying programming %language.
%The layout and bindings to underlying data will be written using Qt's provided %API, and the required libraries will be packaged with the software as allowed %by the license for distribution. 
%\subsubsection{Concerns}
%Users will have a standard interface across platforms, both desktop and mobile,% with due considerations made to the different input methods on desktop and %mobile.
%Software developers will need to develop interface methods that work on %differing platforms instead of a single platform that choosing either desktop %or mobile would allow, creating additional maintenance cost.
%User interface designers accommodating multiple screen resolutions and ratios %may produce an interface that is optimized for neither interface.
%Logic and math
\subsection{Application Logic}
\subsubsection{Approach}
The math driving the calculations is only required to be called on user input events and does not change dynamically throughout the course of its use.
The language used for the application varies on the platform used for the project.
Javascript would be used for a web-based platform, Java would be used for a mobile-based platform, and Python would be used for a desktop-based platform.
The calculation function will only be called when valid inputs are received, and the form of the functions will take a data-driven approach.
Many languages implement features which allow for an optimized and easy-to-read data-driven implementation of functionality.

\subsubsection{Concerns}
Software developers will focus open source development effort in projects that are in languages they prefer, and both Python and Javascript remain popular languages with many enthusiastic followers.
The language chosen will affect the platform versions that are available, while a language such as C++ can run on many platforms without the need to install an additional runtime per computer, a language such as Python is less often used by common users and would require more stringent packaging and installation management to ensure minimal effort between searching for the program to opening it for the first time.

%UI + logic/math
\subsection{Interface Framework Integration}
\subsubsection{Approach}
Bound variables from the user interface will be read by the underlying code.
These inputs will be checked for validity and converted into their proper data types.
Once converted into the types required for the application logic's calculations, the volumetric rate output is calculated and returned to the user.
The interface between the framework and the application logic will be performed using the Model-view-viewmodel design pattern.
\subsubsection{Concerns}
Software developers on open source projects focus their efforts highly on projects where they have the requisite skill.
Model-view-viewmodel is a common design pattern for UI integration in desktop and mobile applications. 
User Interface designers that wish to improve the design will find familiarity with web design frameworks, such as Vue.js, which uses a model-view-viewmodel method to interface between client and server \autocite{mvvm}.
Users will experience a lower performance impact compared to other options, as the application will remain static when not in use.

%Alison Sections
\subsection{Time-based Input Calculations}
\subsubsection{Approach}
This section is important when reducing the math nurses have to do. Instead of asking for total amount administered or number of hours a patient was fed, we could have nurses simply track as they go. Nurses can tell the app at what time they started and stopped and the rate that the feeding was set to. The app should be able to figure out the current time and use past entries to tell nurses what the future rate should be. In this case, nurses wouldn't even need to calculate the number of hours missed, the app should do it. Depending on the platform being used the method can slightly change, however it will always remain fairly standard. We will grab a timestamp of Day:Hour:Minute:Second for calculations. 
\newline
\newline
An few examples of what this may look like include: 
\newline
\newline
Assuming we use the Ionic framework for a mobile interface we would use the date() function to grab a timestamp \autocite{ionicforum}.
\newline
\indent let date = new Date()\newline
\indent console.log("Current Date ",date)
\newline
\newline
Assuming we use JavaScript for a web interface we would use the getTime() function to grab a timestamp.
\newline
\indent Date.getTime()
\newline
\newline
Assuming that we use Cordova framework for a mobile interface we would use the currentdate class to grab the current time data \autocite{cordova}.
\newline
\indent var currentdate = new Date(); \newline
\indent var datetime = currentdate.getHours() + ":"\newline
\indent \indent \indent + currentdate.getMinutes() + ":" \newline
\indent \indent \indent + currentdate.getSeconds();\newline
\subsubsection{Concerns}
We want to ensure that the application is as accurate as possible. Will the device have accurate time? Will the application know if times don't line up properly? Additionally, if times overlap or aren't in the correct order, it must tell the user an incorrect input was entered and reject the input.

\subsection{Data Management}
\subsubsection{Approach}
The app should be able to keep track of data being entered and preserve it until the user clears it for another patient. We may even be able to add the ability to manage multiple patients. The app should also preserve time information so that the timezone and start/stop time don't have to be entered more than once. They should remain static unless the user decides to reset them.
\subsubsection{Concerns}
The most important rules for our data to adhere to include persistence, accuracy, and security. Persistence ensures that the data being defined (such as starting time, total volume, and intervals) continues to exist even when the application is closed and reopened. This can be important when closing the application or even in an emergency such as a power outage. Accuracy and security ensure that the data remains constant unless dictated by the user and that it cannot be changed outside of the application (by user or third party). This ensures the data is not tampered with and does not change between uses. Below are some data storage examples and their respective benefits and concerns.
\newline
\newline
Shared Preferences: 
Great for standard settings, this method of information storage saves strings in key-value pairs. This may actually be sufficient for our application unless we need to start saving objects (which may be possible if we allow for multi-patient management).
\newline
\newline
Internal Storage: 
This method of data storage allows information to persist without being viewed or changed by other applications or the user. This is kept hidden from the user and due to this, may be ideal for our application. We only want the user to be able to interact with user, time, and rate data from within the app.
\newline
\newline
External Storage:
This method of data storage is used when the user wants to export information to view or send someplace else. This method is not idea since the application won't be exporting anything. All data being stored is used by the app and shouldn't be viewable or editable by the user unless within the app. (even then some data is still obfuscated)


\subsection{Interface Design}
\subsubsection{Approach}
This is the section to make the application look pretty. We need to choose a framework that makes the application easy to interact with and enjoyable.
\newline
The obvious solution for any web-based computer application would be HTML and CSS and building off of that with something like JS. However, we may not make a computer application. If this is the case we will need to research some options for mobile apps.
\\\\
\noindent
Mobile Interface Frameworks
\newline
\newline
\noindent
React Native: 
A very popular cross-platform option for Android and iOS development. This framework is inclusive and uses JavaScript. This makes it very familiar, especially for our team who hasn't had experience with mobile app development. This may be the easiest option to learn of the three listed here.
\newline
\newline
Apache Cordova / PhoneGap: 
This framework seems to emphasize the ease of use with peripherals such as the phone's camera and other apps. While this is great, we likely won't need to make use of this. The benefit of this framework is that many versions of an application can be created using the same code base which makes developing for multiple platforms easier.
\newline
\newline
%%%%%%%%
% Commented this out for now, because it probably doesn't apply to our project now that we know it's non-free? Uncomment if I'm wrong on that
%%%%%%%%
%Ionic: 
%This framework comes with a lot of default UI features and integrates Cordova %for easier access to features like the camera and contacts. It allows for %cross-platform development, making it easy to build an app for both Android %and iOS. This could be a really good candidate for our project due to it's %inclusive nature. 
%\newline
%\newline Edit: After more research I found that publishing with Ionic is not %free while React Native is. 
%\newline
%\newline
\noindent
Desktop Interface Framework
\\\\
Qt: This framework includes many built-in features and covers a wide variety of platforms, including macOS, windows, and linux \autocite{qplatforms}. 
The framework is also widely used in the open-source community and has support for a variety of languages, including C++, Java, and Python \autocite{qplatforms}.
The wide user base additionally provides a wealth of documentation and example projects from which to draw inspiration.

\subsubsection{Concerns}
Note: Biggest issue is the uncertainty of our platform. This will be revised for the final draft!!!!!!!! When this is done, the above section will be revised to have just the singular option chosen.
\newline
For now, the main concerns to consider are: Does it cost money? Is what we want to do feasible on this platform? All the options listed above are viable, the main roadblock is choosing what the nurses prefer.
%=======================================================
\section{Conclusion}
The enteral feeding calculator will first and foremost be an application to quicken and reduce the time of administration time for nurses and quicken responses for dieticians, but the calculator solves a variety of other issues as well.
The calculator will offer nurses, its users, a simple and reliable interface to produce their needed values.
The design  and open source nature of the process will provide administrators, providers, and dieticians an easy to justify and available product.
The architecture of the project provides future developers and contributors an easy start on upkeeping the project after its initial creation to ensure the application continues to improve.
Utilizing a UI framework for the interface of the project makes an app that is visually consistent with many other applications that also use it in addition to reducing development time and improving maintainability.
All of these aspects lend themselves to a enteral feeding rate calculator that is simple, fast, accurate, and a proper replacement for paper tables.
%=======================================================
\printbibliography[title=\section{References}]
%\section{References}
%
%“How to Store Data Locally in an Android App.” Android Authority, 20 Nov. 2017, www.androidauthority.com/how-to-store-data-locally-in-android-app-717190/.
%\newline
%\newline
%“Top 10 Best Mobile App Development Frameworks in 2019–20.” By Alex Hales, hackernoon.com/top-10-best-mobile-app-development-frameworks-in-2019-612b95cf930f.
%\newline
%\newline
%Krishna12. “How to GetCurrent Time without Using Picker.” Ionic Forum, 15 Nov. 2017, forum.ionicframework.com/t/how-to-getcurrent-time-without-using-picker/112208.
%\newline
%\newline
%Zak-DevZak-Dev. “Cordova : Android Device Current Date/Time.” Stack Overflow, 1 Jan. 1968, stackoverflow.com/questions/46781636/cordova-android-device-current-date-time.
%\\
%\\
% Martin Fowler. Presentation Model. 2004. url: https://martinfowler.com/eaaDev/PresentationModel.html (visited on 11/03/2019).
% \\
% \\
% Naveed Khan. Introduction to Data Driven Programming. 2018. url: https://www.effective-programmer.com/2018/05/27/introduction-to-data-driven-programming/ (visited on 11/08/2019.
% \\
% \\
%  Mike Potel. The Taligent Programming Model for C++ and Java. 1996. url: http://www.wildcrest.com/Potel/Portfolio/mvp.pdf (visited on 11/03/2019).
%\\
%\\
%The Qt Company. Supported Platforms. 2019. url: https://doc.qt.io/qt-5/supported-platforms.html (visited on 11/03/2019).
%\\
%\\
%FDA. U. S. Food \& Drug Administration. 2019. url: https://www.fda.gov/media/80958/download (visited on 11/19/2019).
\end{document}
