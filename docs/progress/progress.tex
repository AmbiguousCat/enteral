\documentclass[onecolumn, draftclsnofoot,10pt, compsoc]{IEEEtran}
\usepackage{graphicx}
\usepackage{url}
\usepackage{setspace}
% Set backend as appropriate
\usepackage[style=ieee,backend=biber]{biblatex}
\usepackage{geometry}
\bibliography{references}
\geometry{textheight=9.5in, textwidth=7in}

\def \CapstoneTeamName{		Enteral Feeding Calculator Team}
\def \CapstoneTeamNumber{		60}
\def \GroupMemberOne{			Alison Jones}
\def \GroupMemberTwo{			Parker Okonek}
\def \CapstoneProjectName{		Volume-Based Enteral Feeding Calculator}
%\def \CapstoneSponsorCompany{	Company, Inc}
\def \CapstoneSponsorPerson{		Dr. Judy Davidson}

\def \DocType{
				Final Progress Report
				}
			
\newcommand{\NameSigPair}[1]{\par
\makebox[2.75in][r]{#1} \hfil 	\makebox[3.25in]{\makebox[2.25in]{\hrulefill} \hfill		\makebox[.75in]{\hrulefill}}
\par\vspace{-12pt} \textit{\tiny\noindent
\makebox[2.75in]{} \hfil		\makebox[3.25in]{\makebox[2.25in][r]{Signature} \hfill	\makebox[.75in][r]{Date}}}}
% 3. If the document is not to be signed, uncomment the RENEWcommand below
\renewcommand{\NameSigPair}[1]{#1}

%%%%%%%%%%%%%%%%%%%%%%%%%%%%%%%%%%%%%%%
\begin{document}
\begin{titlepage}
    \pagenumbering{gobble}
    \begin{singlespace}
        \hfill 
        \par\vspace{.2in}
        \centering
        \scshape{
            \huge CS Capstone \DocType \par
            {\large\today}\par
            \vspace{.5in}
            \textbf{\Huge\CapstoneProjectName}\par
            \vfill
            {\large Prepared for}\par
            {\Large\NameSigPair{\CapstoneSponsorPerson}\par}
            {\large Prepared by }\par
            Group\CapstoneTeamNumber\par
            \CapstoneTeamName\par 
            \vspace{5pt}
            {\Large
                \NameSigPair{\GroupMemberOne}\par
                \NameSigPair{\GroupMemberTwo}\par
            }
            \vspace{20pt}
        }
        \begin{abstract}%temp abstract until doc is complete.
        Abstract goes here.
        \end{abstract}     
    \end{singlespace}
\end{titlepage}
\newpage
\pagenumbering{arabic}
%\tableofcontents
\clearpage
\section{Overview}
The enteral feeding calculator aims to provide simple, user-oriented interface for determining catch-up feeding rates for missed time in clinical tube feeding.
This method will be focused on ease of use for nurses and dieticians and remove the extra steps in calculation when done by hand or done by the current paper table.
Goals of this project, with success determined by surveying a test group of nurses, include: a 70\% reduction in feeding rate errors, a minimum 50\% increase in nurse satisfaction, and an increase in perceived availability of the program over the paper method.
\section{Status}
The project's documentation is completed and updated with regards to commentary by the client.
No work has been done on the code base of the project, however there have been methods, UI libraries, and languages selected for the eventual platform of the project.
The platform of the project is not yet selected, and a user study is still being arranged to be done sometime before the end of second week of winter term.
\section{Problems}
Our team was not able to secure a set of nurses to interview for determining a platform before the client verification deadline, we have discussed this with our client and are currently working to resolve any platform issues before serious implementation must begin.
\section{Retrospective}
\begin{tabular}{p{0.3\linewidth} p{0.3\linewidth} p{0.3\linewidth}}
\large{Positives} & \large{Deltas} & \large{Actions}\\
\hline
Weeks 0-2 & \nobreakspace & \nobreakspace\\
\hline
Stuff & that & happened\\
\hline
Weeks 3-4 \nobreakspace & \nobreakspace\\
\hline
Stuff & we & did\\
\hline
Weeks 5-6 \nobreakspace & \nobreakspace\\
\hline
para & graphs & here\\
\hline
Weeks 7-8 \nobreakspace & \nobreakspace\\
\hline
para & graphs & here\\
\hline
Weeks 9-10 \nobreakspace & \nobreakspace\\
\hline
para & graphs & here\\
\hline


\end{tabular}
\end{document}