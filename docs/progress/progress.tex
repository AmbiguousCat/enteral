\documentclass[onecolumn, draftclsnofoot,10pt, compsoc]{IEEEtran}
\usepackage{graphicx}
\usepackage{url}
\usepackage{setspace}
% Set backend as appropriate
\usepackage[style=ieee,backend=biber]{biblatex}
\usepackage{geometry}
\bibliography{references}
\geometry{textheight=9.5in, textwidth=7in}

\def \CapstoneTeamName{		Enteral Feeding Calculator Team}
\def \CapstoneTeamNumber{		60}
\def \GroupMemberTwo{			Alison Jones}
\def \GroupMemberOne{			Parker Okonek}
\def \CapstoneProjectName{		Volume-Based Enteral Feeding Calculator}
%\def \CapstoneSponsorCompany{	Company, Inc}
\def \CapstoneSponsorPerson{		Dr. Judy Davidson}

\def \DocType{
				Final Progress Report
				}
			
\newcommand{\NameSigPair}[1]{\par
\makebox[2.75in][r]{#1} \hfil 	\makebox[3.25in]{\makebox[2.25in]{\hrulefill} \hfill		\makebox[.75in]{\hrulefill}}
\par\vspace{-12pt} \textit{\tiny\noindent
\makebox[2.75in]{} \hfil		\makebox[3.25in]{\makebox[2.25in][r]{Signature} \hfill	\makebox[.75in][r]{Date}}}}
% 3. If the document is not to be signed, uncomment the RENEWcommand below
\renewcommand{\NameSigPair}[1]{#1}

%%%%%%%%%%%%%%%%%%%%%%%%%%%%%%%%%%%%%%%
\begin{document}
\begin{titlepage}
    \pagenumbering{gobble}
    \begin{singlespace}
        \hfill 
        \par\vspace{.2in}
        \centering
        \scshape{
            \huge CS Capstone \DocType \par
            {\large\today}\par
            \vspace{.5in}
            \textbf{\Huge\CapstoneProjectName}\par
            \vfill
            %{\large Prepared for}\par
            %{\Large\NameSigPair{\CapstoneSponsorPerson}\par}
            {\large Prepared by }\par
            Group\CapstoneTeamNumber\par
            \CapstoneTeamName\par 
            \vspace{5pt}
            {\Large
                \NameSigPair{\GroupMemberOne}\par
                \NameSigPair{\GroupMemberTwo}\par
            }
            \vspace{20pt}
        }
        \begin{abstract}%temp abstract until doc is complete.
        The past ten weeks of work on the enteral feeding project have produced a wealth of documentation and understanding of the core issues surrounding the project.
        The projects main design documents have been created and reviewed by the client of the project, Dr. Judy Davidson, and meetings with her peers and others have shed light on the requirements for them, the developers, nurses, and other end users.
        The platform of the project remained uncertain for most of the development of the project, however this issue among many others was resolved in the last week and a half of this development period.
        \end{abstract}     
    \end{singlespace}
\end{titlepage}
\newpage
\pagenumbering{arabic}
%\tableofcontents
\clearpage
\section{Overview}
The enteral feeding calculator aims to provide simple, user-oriented interface for determining catch-up feeding rates for missed time in clinical tube feeding.
This method will be focused on ease of use for nurses and dieticians and remove the extra steps in calculation when done by hand or done by the current paper table.
%% These numbers are placeholders, but the professors want specific numbers here, so?
Goals of this project, with success determined by surveying a test group of nurses, include: a 70\% reduction in feeding rate errors, a minimum 50\% increase in nurse satisfaction, and an increase in perceived availability of the program over the paper method.

The project's documentation is completed and updated with regards to commentary by the client.
No work has been done on the code base of the project, however there have been methods, UI libraries, and languages selected for the eventual platform of the project.
The platform of the project is not yet selected, and a user study is still being arranged to be done sometime before the end of second week of winter term.
\section{Weekly Summary}
\subsection{Weeks 0-2}
We arranged the first meetings with our client and their project partners where we were introduced and learned the basic outline of the project.
A follow up meeting was also arranged in this time, so we could learn more about the traditional process in use with the paper table and apply that knowledge to the new application.

Individual problem statement drafts were created with our knowledge of the process from the initial meeting, but our knowledge was over-generalized and incorrect.
\subsection{Week 3}
Our team held a follow up meeting with our client about the existing paper process for calculating catch-up tube feeding rates.
In the meeting our team talked with Michael Mercer, who walked through the process with the team by hand and answered our questions from an instructional video on the current table process.
The mathematics behind the application was clarified during this meeting and our team revised and submitted a new group problem statement with this updated and correct information.

Questions about the platform of the application were raised at this time, but delegated to be decided once we had a focus group or focus persons to survey.
\subsection{Week 4}
Our team produced the requirements document, containing information on the minimum requirements of the software.
We determined that the software will require a minimal and clear interface with little to no configuration options, a common platform used in hospitals such as Desktop, and utmost accuracy that is at minimum on par with the paper method and does not exceed safe maximums.

Our team also took part in a meeting with a producer of a paper table currently used by some hospitals, Dr. Heyland, and further clarified the mechanics behind the process.
\subsection{Weeks 5-6}
Our team spend this time producing our individual tech review drafts, which overviewed various prospective technologies to be used in our project.
There was no extra meetings or communication with our client at this time.

\subsection{Weeks 7-8}
Our tech reviews were later used as a basis for the choices of technologies in our design document, where we determined that React Native could be used as a phone interface, and Qt a desktop interface and identified that Android 7.0 or Windows 7 would be our minimum supported software versions.
Our team updated our Gantt chart, used to time and plan division of work with this in mind, and contextualized the project in the scope of its users, which we identified as nurses and dieticians.

\subsection{Weeks 8-9}
The design document was completed and sent to the Judy Davidson, the client, for any required changes in the sake of clarity, accuracy, and addition of missing information.
The documetn was updated after Dr. Davidson's review with a reorganized user scope section, sorted by importance for that user, including dieticians as end users, and adding information about what specifically was outside the scope of the project, among other changes.

\subsection{Week 10}
Our team held a phone meeting with the clinical nutrition manager at the local Good Samaritan Medical Center, Sara Thomas, and discussed from her perspective the needs and preferences of nurses and dieticians on the floor.
Desktop remains a strong contender for the platform of the application and many questions about the input styles the program would accept, such as time ranges vs numbers of hours missed, and how the program will display its information to the user were answered.

\section{Retrospective}
\begin{tabular}{p{0.3\linewidth} p{0.3\linewidth} p{0.3\linewidth}}
\large{Positives} & \large{Deltas} & \large{Actions}\\
\hline
Platform & \nobreakspace & \nobreakspace\\
\hline
The platform was finally determined between desktop and mobile after a multiple weeks of uncertainty in a meeting with Sara Thomas.&
Documentation will need to be updated with the desktop forward approach that our team has adopted.&
The team will update the documents and when revisions are complete send the updated documents to the client for approval.\\
\hline
User Input Types & \nobreakspace & \nobreakspace\\
\hline
We have determined proper input value types for the process and have most of the inputs the users will access chosen for the beginning of development next term.&
We will need to create the application with the ability to accept the inputs in this way and develop the calculating code with this in mind.&
The application will be created to accept numeral time gap inputs (such as 2 hours missing, or 3 hours missing) and prototypes of this application will be reviewed by our client, Sara Thomas, Michael Mercer, and others to confirm these inputs work ergonomically.\\
\hline
UI Design and Layout & \nobreakspace & \nobreakspace\\
\hline
All groups and individuals met with for the project have unanimously agreed on its utility over paper tables and the simple methods of UI design described by our initial talks.&
When we design the UI of the project, we will need to keep simplicity and readability in mind for the brief amount of time nurses are expected to use the application for a single patient and calculation session.&
Our team will be need to create paper prototypes and present them to others to check for fast readability and do extensive usability testing to ensure that the program can be used as it is expected to by our team, our client and their team, and end users.
\end{tabular}
\end{document}