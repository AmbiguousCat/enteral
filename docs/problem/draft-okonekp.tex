\documentclass[10pt,draftclsnofoot,onecolumn]{IEEEtran}
\usepackage[margin=0.75in]{geometry}
\usepackage[backend=biber,citestyle=numeric]{biblatex}
\bibliography{enteral}
%\usepackage{pygments}

\begin{document}
\title{A Volume-based Enteral Feeding Calculator}
\author{Parker Okonek}
\IEEEspecialpapernotice{CS461 Fall 2019}
\maketitle
\begin{abstract}
Enteral feeding determined from volume lacks any automation or computer software to
automatically compute the dose to be administered to critical care patients.
This type of enteral feeding provides higher quality care to patients but has drawbacks for
the administering nurses.

The current method is lengthy and repetitive while also leaving large room for human error.
The process can be improved for patient and nurse outcome by reducing possible error and 
saving the time and effort of critical care nurses who may have multiple patients a day to
administer. A program to allow simple input of the required data and return the dosage would
resolve these problems, but it requires development of a system which is equally accurate
and exists on a platform most nurses are familiar using in a hospital environment.
\end{abstract}
\newpage
\section{Problem}
Enteral feeding is the process of feeding a patient using a feeding tube through the 
mouth or esophagus to ensure proper nutrition, medicine dosing, and patient health while
a patient would be unable to eat normally. 
Volume-based enteral feeding measures the doses of food or drug by using volume instead of
flow rate . It offers an improved success rate for proper drug dosage
 and nutrition absorption  by making recovery for missed or late doses
 much more accurate and easy to determine compared to rate-based methods \autocite{whiteking}.
The volumetric method requires a repetitive, error-prone process performed by hand with
a reference table. This process is time-consuming and frustrating for clinical care nurses which
already have a variety of tasks to perform.  The paper table is determined by a set of other 
calculations which are more time-consuming and error prone to calculate by hand but could
be done quickly and reliably by software.

\section{Solution}
The accuracy provided by the volume-based method can be made more accessible with 
a simple form style interface that accepts inputs from the user on the different metrics required
by the table, including time since last feeding, volume per hour, and other factors. Critical care
nurses input these values into the app to produce the volume for the next feeding session. 
The calculations approximated by the paper table will be performed by the application  and
returned to the nurse in a result field.
Available
critical care nurses will be surveyed through contacts of Dr. Judy Davidson to determine what 
platform is most common and familiar in a modern inpatient care environment. Possible platforms
include mobile devices, iPads and android tablets for example, or a Windows-focused desktop.
The source code for the project will be made available under an open-source license to allow
for community maintenance and less up-front cost for widespread adoption.

\section{Performance Metrics}
The success of the project will be evaluated on its accuracy, ease-of-use, and lack of errors.
The volume based feeding schedule should use the same inputs as the paper table and should 
return equivalent outputs for a set of given inputs. The user interaction should be clear to read
and simple to use for nurses with a variety of technical experience. The program should be 
tested to ensure that it does not crash during normal operation, regardless of time open for normal
use.

Accuracy must be equivalent to the current calculation table.
For example, a patient with a daily volume
goal of 2200 milliliters with 12 hours remaining in the 24 hour period, the table would provide a 
value of 150 milliliters. 
If the application
returns a volume which is incorrect or less precise than the table, it fails to meet the metric for
accuracy. 

Ease-of-use is a primary consideration in an inpatient environment, where there is often a mix of new and old technologies.
The program should have clearly titled, separated inputs with the minimum amount of input fields
possible. The application should also
be resilient against user input errors, with protections against overdosing patients beyond the
highest possible safe volume of food or drug.

Stability in the application and removal of non-crashing errors will be required to continued use.
The application could be left open for long periods or closed and re-opened multiple times 
over the course of a twenty four hour time slot. The program should be equally usable in both 
of these cases without significant slowdown or corruption of output data. Freezing in the application
such that it cannot be terminated should be wholly prevented.

The automation of volume-based enteral feeding calculations prevents the need for tabulating
the results manually on a table, thereby reducing human error, allowing instant feedback on
the quality of return values, reducing effort of critical care nurses, and allowing nurses to offer
care more quickly and effectively.

\newpage
\printbibliography
\end{document}