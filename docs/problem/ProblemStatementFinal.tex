\documentclass[fullpage,10pt, onecolumn, draftclsnofoot]{IEEEtran}
\usepackage[utf8]{inputenc}
\usepackage[margin=.75in]{geometry}
\usepackage{setspace}
\singlespacing

\begin{document}
\pagenumbering{arabic}
\title{Patient Safety App: A Volume-based Enteral Feeding Calculator}
\author{Alison Jones \& Parker Okonek}
 \IEEEspecialpapernotice{CS461 - Senior Project \\Fall 2019}
 \maketitle
 \begin{abstract}
 Current clinical technology is often outdated. While medicine and doctors have improved their 
effectiveness over the past few decades and technology has reached amazing capabilities, 
many tools used in a clinical environment lack any automation or computer software to
accompany them.  Enteral feeding, a frequently used medical procedure, is an example of 
this disjoint relationship. The current method for determining feeding rates is lengthy and repetitive
while also leaving large room for human error. 

The calculations being made
can easily be automated and presented to nurses in the form of an app. It is our team's job to
develop this application and test it in both simulated and clinical environments. The application
should take minimal input from nurses and provide them with the correct feeding rate and
be released with open source to aid in deployment.

After development, the app will be taken into clinical
 trials to test it's efficacy over the original hand-calculated method. To conduct trials our team
 must verify that we follow the rules outlined by the FDA and IRB.
 \end{abstract}
\newpage
\section{Problem Description}
Enteral feeding is the process of delivering  liquid  food through a tube into the 
stomach or intestine to ensure proper nutrition, medicine dosing, and patient health while
a patient would be unable to eat normally. The volumetric method requires a repetitive, error-prone process performed
by hand with a reference table to calculate the catch-up rate for the tube feeding. There are four different steps taken that could easily be
automated. Human error is inevitable with this method and puts patients at an unnecessary risk. 

The paper table is determined by a set of other 
calculations which are more time-consuming and error prone to calculate by hand but could
be done quickly and reliably by software. Errors made when using this method are known to exist, however the actual frequency is unknown.
We not only have errors being made, but also don't understand the full extent of these errors. 
\section{Proposed Solution}
The app will automate as many steps as possible for the nurses. Less human 
 interaction with a system known to work should reduce error. 
The accuracy provided by the volume-based method can then be made more accessible with 
a simple form style interface that accepts inputs from the user on the different inputs required
by the table, including time since last feeding, volume per hour, and other factors.

The math driving the calculations of the program will be plain algebra; however, it should be wrapped into an application built
for the most desirable interface (web, phone, tablet, etc.). The interface must also be intuitive for
critical care nurses with a variety of backgrounds and technological expertise or lack thereof.
Available
critical care nurses will be surveyed through contacts of Dr. Judy Davidson to determine what 
platform is most common and familiar in a modern inpatient care environment.
To verify that the proposed solution
works clinical trials should be conducted with the new method and compared to data from trials 
that use the outdated method.

\section{Performance Metrics}
The success of the project will be evaluated on its accuracy, ease-of-use, and lack of errors.
In addition, there will be checkpoints to denote major completed steps of the project.
The app should have user input as small as possible while still generating correct results.
It should also correctly calculate feeding rate for patients. To do this the team will get firsthand 
training on how nurses go through the process. The team should test the app using simulated data 
to verify that it is safe and reliable for use in a hospital environment. 

Ease-of-use is a primary consideration in an inpatient environment, where there is often a mix of new and old technologies.
The program should have clearly titled, separated inputs with the minimum amount of input fields
possible. The application should also
be resilient against user input errors, with protections against overdosing patients beyond the
highest possible safe volume of food or drug.

The app will be released open source. The technology should be available so hospitals
have access to a possibly safer alternative to the outdated method. This is to ensure that the app
helps as many people as possible. 

Stability in the application and removal of non-crashing errors will be required to continued use.
The application could be left open for long periods or closed and re-opened multiple times 
over the course of a twenty four hour time slot. The program should be equally usable in both 
of these cases without significant slowdown or corruption of output data. Freezing in the application
such that it cannot be terminated should be wholly prevented.

The automation of volume-based enteral feeding calculations prevents the need for tabulating
the results manually on a table, thereby reducing human error, allowing instant feedback on
the quality of return values, reducing effort of critical care nurses, and allowing nurses to offer
care more quickly and effectively.

As a stretch goal, the team should bring the app into the real world for clinical testing. 
In order to do this, we must get permission and IRB oversight. This paperwork can take months to 
process and must be started early. Additionally, we must make sure that we comply with any rules
outlined by the FDA. The point of this goal is to gather data from nurses and patients and show 
whether or not the project has successfully reduced error and improved patient safety and outcome 
compared to the old method. 
 \end{document}