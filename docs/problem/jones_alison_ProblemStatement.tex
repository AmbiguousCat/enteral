\documentclass[fullpage,10pt, onecolumn, draftclsnofoot]{IEEEtran}
\usepackage[utf8]{inputenc}
\usepackage[margin=.75in]{geometry}
\usepackage{setspace}
\singlespacing

\title{CS60 Patient Safety App: A Volume-based Enteral Feeding Calculator}
\author{Alison Jones\\ CS461 - Senior Project \\Fall 2019}

\begin{document}
\pagenumbering{arabic}
\maketitle

\section{abstract}
Current clinical technology is often outdated. While medicine and doctors have improved their effectiveness over the past few decades and technology has reached amazing capabilities, the two don't always come together. A frequently used medical procedure known as enteral feeding which involves administering liquid food through a tube is an example of this disjoint between medicine and technology. Nurses have to hand calculate the rate of food administration. This human involvement can result in error, compromising patient safety. The calculations being made can easily be automated and presented to nurses in the form of an app. It is our team's job to develop this application and test it in both simulated and clinical environments. The application should take minimal input from nurses and provide them with the correct feeding rate; ultimately reducing human interaction. After development, the app will be taken into clinical trials to test it's efficacy over the original hand-calculated method. To conduct trials our team must verify that we follow the rules outlined by the FDA and IRB. Additionally, the project will be open source so it can be made available for anyone. 

\newpage

\section{Problem Description}
Technology in the healthcare world is often outdated or nonexistent. While our healthcare has greatly improved over the past few decades, we could be doing better! There are errors being made that could easily be avoided by integrating cutting-edge technology. This would likely reduce human error and increase the safety of patients.
\newline Many patients in the hospital require tube feeding. Liquid food is delivered through a tube into the stomach or intestine. This is also referred to as enteral feeding. Nurses have several steps to go through when calculating the rate to use for treatment. This raises the potential for error, and can  prevent patients from getting the nutrition they need. The current, method involves nurses using a paper table to calculate the rate/catch-up rate for the tube feeding. There are four different steps taken that could easily be automated. Human error is inevitable with this method and puts patients at an unnecessary risk. 
\newline Errors made when using this method are known to exist, however the actual frequency is unknown. We not only have errors being made, but also don't understand the full extent of these errors. 


\section{Proposed Solution}
The team will learn the current method being used by nurses, understand the math behind the feeding table, and turn it into an app. The app will automate as many steps as possible for the nurses. Less human interaction with a system known to work should reduce error. 
\newline The app has four steps that we will need to program. The math portion of the code should be fairly straightforward. After coding the process, it should be wrapped into an application built for the most desirable interface (web, phone, tablet, etc.). The app should not only work, it should also be intuitive. This is meant to make the nurse's job easier! To verify that the proposed solution works clinical trials should be conducted with the new method and compared to data from trials that use the outdated method.

\section{Performance Metrics}
The team has a few major checkpoints in order to complete the project. The first metric is having a completed app to aid nurses with feeding calculations. This app can be on the phone, web, or computer. Our team must make a decision on which to use based on what will be most convenient, efficient, or desirable for the target user (nurses). The second metric is testing the completed app for efficiency, effectiveness, and broader use. 
\newline The app should have user input as small as possible while still generating correct results. It should also correctly calculate feeding rate for patients. To do this the team will get firsthand training on how nurses go through the process. The team should test the app using simulated data to verify that it is safe and reliable for use in a hospital environment. 
\newline The app should also be open source! The technology should be available so hospitals have access to a possibly safer alternative to the outdated method. This is to ensure that the app helps as many people as possible. 
\newline As a stretch goal, the team should bring the app into the real world for clinical testing. In order to do this, we must get permission and IRB oversight. This paperwork can take months to process and must be started early. Additionally, we must make sure that we comply with any rules outlined by the FDA. The point of this goal is to gather data from nurses and patients and show whether or not the project has successfully reduced error and improved patient safety and outcome compared to the old method. 

\end{document}
