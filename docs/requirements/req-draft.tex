\documentclass[10pt,draftclsnofoot,onecolumn]{IEEEtran}
\usepackage[margin=0.75in]{geometry}
\usepackage[backend=biber,citestyle=numeric]{biblatex}
\usepackage{listings}

\begin{document}
\title{}
\author{Alison Jones \& Parker Okonek}
\IEEEspecialpapernotice{CS461 Fall 2019}
\IEEEspecialpapernotice{Group Name}
\maketitle
\begin{abstract}
%Abstract Goes here
\end{abstract}
\newpage
%Body text goes here
\section{Introduction}
\subsection{Purpose}
\subsection{Scope}
\subsection{Product Overview}
\subsubsection{Product Perspective}
\subsubsection{Product Functions}
\subsubsection{User Characteristics}
\subsection{Definitions}

%may be optional for us
\section{References}

\section{Overall Description}
\subsection{Product Perspective}
- open source
- application or web interface

\subsection{Product Functions}
- takes in time input
- calculates volumetric feeding rate 

\section{Specific Requirements}
\subsection{Hardware Interfaces}
- Storage for program config
- Display

\subsection{User Interfaces}
- phone/desktop application or web app.
- simple/ minimal user entry
- time and text entry/ possibly int entry

\subsection{Functional Requirements}
- must do THE math
- need to be able to get date/time info
- update variables
- time start/end should be preserved after set. Can be reset.
- data should be stored locally.

\subsection{Non-Functional Requirements}
\subsubsection{}
\subsubsection{Software Quality}
- Needs to be able to run on a variety of platforms
- doesn't crash
- returns numbers the same for every input

\section{Verification}

\end{document}
