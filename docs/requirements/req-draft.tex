\documentclass[10pt,draftclsnofoot,onecolumn]{IEEEtran}
\usepackage[margin=0.75in]{geometry}
\usepackage[backend=biber,citestyle=numeric]{biblatex}
\usepackage{listings}

\begin{document}
\title{}
\author{Alison Jones \& Parker Okonek}
\IEEEspecialpapernotice{CS461 Fall 2019}
\IEEEspecialpapernotice{Group Name}
\maketitle
\begin{abstract}
%Abstract Goes here
\end{abstract}
\newpage
%Body text goes here
\section{Introduction}

\subsection{Purpose}
The volume-based enteral feeding calculator replaces the paper tables currently used when 
calculating feed rates for both gastric and small bowel enteral feeding. The calculator reduces
error by requiring the source numbers to be entered and automatically produced, unlike the current
method which requires the result to be produced by hand. The goal of this conversion is to reduce both
error and time taken in getting the results.

\subsection{Scope}

\subsection{References}

\subsection{Product Overview}
The application and its source code will be released on GitHub. Both the compiled binary and the source code
will be appropriately licensed for its own open source distribution and to be license compatible with any 
libraries, code, or other intellectual property used to create the application.

The platform for the project will be determined from surveying nurses to determine whether mobile or desktop
is more convenient and widespread for projects of similar scale and niche.

\subsection{Product Functions}
- takes in time input
- calculates volumetric feeding rate

\section{Specific Requirements}
\subsection{System Interfaces}
The enteral feeding calculator will interface with local storage on the machine to store any generated 
error logs and to save values which are not likely to change after being first set. The calculator will
also display to the user's screen and input devices via a UI library to interface with the operating system's screen drawing
routines and to accept pointer and text input from the user.

\subsection{User Interfaces}
The user interface will be minimal and easy to read. The application will have no configuration window
or views other than its main interface, and on its main interface will have the text inputs for the different
input values: time of feeding stop, time of feeding restart, reset time of feeding, and daily feeding volume.

Time inputs will allow only numerical values, a colon, and optionally AM/PM inputs. Any time with a PM will
have 12 added to its value automatically before being used for calculations. A time with PM at the end cannot have an hour value greater than 12.
Volume inputs will accept only numerical values. A checkbox will be used to indicate whether the feeding type is gastric or
small bowel, with a large banner indicating the current state. At the bottom of the UI will be the output value and a calculation
button that updates the output value, if valid.

\subsubsection{Functions}
- must do THE math
- need to be able to get date/time info
- update variables
- time start/end should be preserved after set. Can be reset.
- data should be stored locally.

\subsubsection{Software Quality}
- Needs to be able to run on a variety of platforms
- doesn't crash
- returns numbers the same for every input

\section{Verification}

\end{document}
